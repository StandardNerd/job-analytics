\documentclass{article}
\usepackage[utf8]{inputenc}
\usepackage[ngerman]{babel}

\title{Exposé: Ermittlung der Anforderung digitaler Kompetenzen von Mitarbeitenden in öffentlichen Verwaltungen mittels automatischer Auswertung von Stellenanzeigen}
\author{Joon Ki Choi}
\date{\today}

\begin{document}

\maketitle

\section{Forschungsfrage und Zielsetzung}

\hspace{11pt} \textbf{Zentrale Forschungsfrage:} "Was sind die digitalen Kompetenzen, die an

\setlength{\parindent}{15pt}Mitarbeitende in öffentlichen Verwaltungen heutzutage gestellt werden?"\\

\textbf{Ziele der Arbeit:}

\begin{itemize}
	\item Entwicklung eines automatisierten Systems zur Analyse von Stellenanzeigen für die Identifikation und Kategorisierung digitaler Kompetenzanforderungen
	\item Schaffung einer wissenschaftlich fundierten Übersicht aktueller digitaler Kompetenzprofile nach DigComp Framework
\end{itemize}

\section{Methodischer Ansatz}

\subsection{Bestimmung von Digitalen Kompetenzen}
Beschreiben und Ermitteln der digitalen Kompetenzen nach DigComp Framework.

\subsection{Entwicklung der technischen Komponenten}
\begin{itemize}
	\item Entwicklung eines Web-Scrapers zur Sammlung von Stellenanzeigen (in öffentlichen Verwaltungen)
	\item Integration von Large Language Models (LLMs) zur automatisierten Datenextraktion und -kategorisierung
	\item Entwicklung einer Webapplikation zur Systemsteuerung und Ergebnispräsentation
\end{itemize}

\subsection{Methodische Schritte}
\begin{itemize}
	\item Anforderungsanalyse und Systemdesign
	\item Aufbau eines leistungsfähigen AI-Servers zum lokalen Betrieb von LLMs
	\item Vergleich kommerzieller und Open-Source LLMs
	\item Implementation und iterative Validierung des Systems
\end{itemize}

\section{Erwartete Ergebnisse}
\begin{itemize}
	\item Applikation des entwickelten Analysesystems als Opensource Software
	\item Erkenntnisse zum aktuellen Stand zur digitalen Kompetenzanforderung in öffentlichen Verwaltungen
\end{itemize}

\section{Technische Spezifikationen}
\textbf{Technologiestack:}
\begin{itemize}
	\item Web-Scraping: Ruby, Watir (Selenium)
	\item LLM-Integration: *to be defined*
	\item Webapplikation: Ruby on Rails, JS Client
	\item Datenbank: PostgreSQL
	\item Serverinfrastruktur: Proxmox, AI-Server mit GPU-Unterstützung
\end{itemize}

\clearpage

\section{Zeitplan}
\textbf{Meilensteine:}
\begin{itemize}
	\item \textbf{Konzeptphase: 2 Wochen}
	      \begin{itemize}
		      \item Recherche
		      \item Installation und Konfiguration AI-Server
		      \item Anwendungs-Systemarchitektur
		      \item Betrieb und Vergleichstests LLMs
	      \end{itemize}
	\item \textbf{Entwicklungsphase: 8 Wochen}
	      \begin{itemize}
		      \item Scraper-Entwicklung
		      \item LLM-Integration
		      \item Webapplikation (Client/Server)
	      \end{itemize}
	\item \textbf{Validierungsphase: 1 Woche}
	      \begin{itemize}
		      \item Systemtests
		      \item Datenanalyse
		      \item Optimierung
	      \end{itemize}
	\item \textbf{Dokumentationsphase: 1 Woche}
	      \begin{itemize}
		      \item Wissenschaftliche Ausarbeitung
		      \item Technische Dokumentation
	      \end{itemize}
\end{itemize}

\section{Erwartete Herausforderungen}
\begin{itemize}
	\item Qualität und Konsistenz der Auswertungsergebnisse von LLMs
	\item Technische Komplexität der Systemintegration und Automatisierung (Automatisierungsgrad, Stabilität/Robustheit, Performance)
\end{itemize}

\end{document}
